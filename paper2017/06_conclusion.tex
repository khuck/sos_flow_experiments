
%%%%%%%%%%%%%%%%%%%%%%%%%%%%%%%%%%%%%%%%%%%%%%%%%%%%%%%%%%%%%%%%%%%%%%%%%%%%%%%
%%%%%%%%%%%%%%%%%%%%%%%%%%%%%%%%%%%%%%%%%%%%%%%%%%%%%%%%%%%%%%%%%%%%%%%%%%%%%%%
%%%%%%%%%%%%%%%%%%%%%%%%%%%%%%%%%%%%%%%%%%%%%%%%%%%%%%%%%%%%%%%%%%%%%%%%%%%%%%%
\section{Conclusion}
%
%  Reference the claims made in the abstract and Intro, make sure we
%  satisfied them properly with the experiments.
%

\subsection{Future Work}
%\textbf{Development}: Continue refining and expanding the core SOSflow
%libraries and the SOS workflow performance model.
%\\
%\textbf{Optimization}: Optimize the SOSflow codes for memory use
%and data latency. Add mechanisms for throttling of data flow to
%increase reliability in resource-constrained cases. Map out 
%best-fit metrics for dedicating in situ resources to a monitoring
%platform for the major extant and proposed compute clusters, and build this
%intelligence into the SOSflow implementation.
%\\
%\textbf{Integration}: Explore options for deployment and integration with
%existing HPC monitoring and analytics codes at LLNL and other
%national laboratories.
%


%%%%%%%%%%%%%%%%%%%%%%%%%%%%%%%%%%%%%%%%%%%%%%%%%%%%%%%%%%%%%%%%%%%%%%%%%%%%%%%
%%%%%%%%%%%%%%%%%%%%%%%%%%%%%%%%%%%%%%%%%%%%%%%%%%%%%%%%%%%%%%%%%%%%%%%%%%%%%%%
%%%%%%%%%%%%%%%%%%%%%%%%%%%%%%%%%%%%%%%%%%%%%%%%%%%%%%%%%%%%%%%%%%%%%%%%%%%%%%%
\subsection{Acknowledgments}

The research report was supported by a grant (DE-SC0012381) from the
Department of Energy, Scientific Data Management, Analytics, and
Visualization (SDMAV), for ``Performance Understanding and Analysis
for Exascale Data Management Workflows.''


%%%
%%%  EOF
%%%
