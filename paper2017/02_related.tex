
%%%%%%%%%%%%%%%%%%%%%%%%%%%%%%%%%%%%%%%%%%%%%%%%%%%%%%%%%%%%%%%%%%%%%%%%%%%%%%%
%%%%%%%%%%%%%%%%%%%%%%%%%%%%%%%%%%%%%%%%%%%%%%%%%%%%%%%%%%%%%%%%%%%%%%%%%%%%%%%
\section{Related Work}
%
Monitoring systems can generally be separated by their purpose into three
independent sets: system, application, and science.
%
System-oriented monitoring focuses on machine metrics such as bandwidth,
queue depth, resource contention, power, and thermal measurements.
%
Exclusively system-oriented monitoring platforms primarily serve the
needs of machine and site administrators, reporting the health and utilization
of the nodes, network, and storage systems.
%
Application-oriented monitoring looks at the performance of software components,
exploring both single-process performance characteristics as well as behaviors
that emerge in parallel distributed contexts.
%
These tools are often of use to low-level application developers and library
writers, but traditionally not helpful to administrators or end users.
%
Science-oriented monitoring involves processing information related to the
simulation phenomenon itself.
%
This information can be used for anomaly or convergence detection,
visualizations, validation of accuracy, and many other domain-specific ends.
%
While science-related information is important to end-users, it has in the
past not been of significant value to developers or administrators.
%
This is no longer true.
%
\par
\todo[inline]{REFER TO PAPERS ON PRECISION, POWER vs. PERFORMANCE,
REPLICATION, ETC. THE KEY IS THAT THESE MODERN EFFORTS OFTEN REQUIRE
DATA THAT IS NORMALLY ONLY GATHERED FOR DISJOINT PURPOSES OR BY
UNRELATED TOOLKITS}
\par
%
%To remain performant at scale, both system and application-oriented
%monitoring frameworks are often limited to predefined purposes such as
%the aggregation of performance tool instrumentation, efforts to conserve power,
%load balancing, or run-time solver selection.
%
SOSflow is designed to address a variety of purposes for monitoring platforms,
including those described above.
%
By integrating various layers and categories of information, operating in
situ adjacent applications at run time, and supporting online query,
analysis, and feedback, SOSflow is able to facilitate both current and
future HPC needs.
%
The following sections describe existing and related HPC monitoring efforts,
contrasted with SOSflow.
%
Though there is inevitably some overlap, projects are grouped into
the category that the they primarly serve.
%
\subsection{System Monitoring}
%
...
%
\subsection{Application Monitoring}
%
...
%
\subsection{Science Monitoring}
%
...
%
\par
%
To the authors' knowledge, no application-oriented online monitoring systems
exist that are feasible for deployment to modern architectures at large scale
and can provide runtime interactivity, are programmable, and capable of
serving purposes beyond simple migration of performance metrics.
%
SOSflow makes a novel contribution by synthesizing application and system
monitoring perspectives into a holistic architecture.
%
SOSflow goes much further than merely interleaving two data sources, as it:
%
\begin{itemize}
    \item Operates during the application runtime
    \item Allows for local and global interactivity
    \item Couples user-defined analytics with applications
    \item Automatically archives data for offline analysis
    \item Provides full-support for complex, loosely-coupled,
        multi-component scientific workflows, as well as simple
        single-binary applications        
    \item Enables online query and feedback to both applications
        and their execution environment
    \item Is engineered to scale from single-node deployments
        to future-scale clusters
\end{itemize}
%
SOSflow is a fully-programmable system designed to be tailored to individual
use-cases with a minimuim amount of additional development or effort.
%
%%%%%%%%%%%%%%%%%%%%%%%%%%%%%%%%%%%%%%%%%%%%%%%%%%%%%%%%%%%%%%%%%%%%%%%%%%%%%%%
%%%%%%%%%%%%%%%%%%%%%%%%%%%%%%%%%%%%%%%%%%%%%%%%%%%%%%%%%%%%%%%%%%%%%%%%%%%%%%%
%%%
%%%  EOF
%%%
