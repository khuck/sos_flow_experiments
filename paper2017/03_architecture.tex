%%%%%%%%%%%%%%%%%%%%%%%%%%%%%%%%%%%%%%%%%%%%%%%%%%%%%%%%%%%%%%%%%%%%%%%%%%%%%%%
%%%%%%%%%%%%%%%%%%%%%%%%%%%%%%%%%%%%%%%%%%%%%%%%%%%%%%%%%%%%%%%%%%%%%%%%%%%%%%%
\section{The Architecture of SOSflow}
%
To better understand what can be done with SOSflow, it is useful to examine
its architecture.
%
SOSflow is composed of four major components:
\begin{itemize}
\item \textbf{sosd} : Daemons
\item \textbf{libsos} : Client Library
\item \textbf{pub/sql} : Data
\item \textbf{sosa} : Analytics \& Feedback 
\end{itemize}
%
These components work together to provide extensive runtime capabilities to
developers, administrators, and application end-users.
%
All of SOSflow runs in user-space, and does not require elevated priveledges
to use any of its features.
%
%
% ----------------------------------------------------------------------------%
\subsection{SOSflow Daemons}
%
Online functionality of SOSflow is enabled by the presence of a
user-space daemon.
%
This daemon operates completely independently from any applications, and
does not interfere with or utilize any of their resources or
communications.
%
The SOSflow daemons are launched from within a job script, before the user's
applications are initialized.
%
These daemons discover and communicate amongst each other across 
node boundaries within a user's allocation.
%
When crossing node boundaries, SOSflow uses the machine's high-speed
communication fabric.
%
Inter-node communication may use either \textbf{MPI} or \textbf{EVPath} as
needed, allowing for flexibility when configuring its deployment to
various HPC environments.
%
\begin{figure}[h]
\centering
\includegraphics[width=\columnwidth]{images/placeholder.jpg}
\caption{SOSflow in situ and off-node daemons.}
\label{fig_daemons}
\end{figure}
%
%
\par
%
The traditional deployment of SOSflow will have a single daemon instance
running in situ for each node that a user's applications will be
executing on.
%
Additional resources can be allocated in support of the SOSflow runtime
as-needed to support scaling and to minimize perturbation of application
performance.
%
One or more nodes are usually added to the user's allocation to
host SOSflow aggregation daemons that combine the information
that is being collected from the in situ daemons.
%
These aggregator daemons are useful for providing a holistic unified view
at runtime, especially in service to online analytics modules.
%
Because they have more work to do than the in situ listener daemons, and
also are a useful place to host analytics modules, it is advisable to
place them on their own dedicated node[s].
%
%
\subsubsection{In Situ} %----------%
%
Data coming from SOSflow clients moves into the in situ daemon across a
light-weight local socket connection.
%
Any software that connects in to the SOSflow runtime can be thought of as a
client.
%
Clients connect only to the daemon that is running on their same node.
%
No socket connections are made across node boundaries, so no special
permissions are required to use SOSflow.
%
%
\par
The in situ daemon offers the complete functionality of the SOSflow runtime.
%
At startup, it creates a local persistent data store that gathers in all
data that is submitted to it.
%
This local data store can be queried and updated via the SOSflow API, with
all information moving over the local socket, avoiding any dependence on
or contention with the filesystem resources.
%
Providing the full spectrum of data collected on node to clients and
analytics modules on node allows for distributed online analytics
processing.
%
Analytics modules running in situ can observe a manageable data set, and then
exchange small intermediate results amongst themselves in order to compute
a final global view.
%
SOSflow also supports running analytics at the aggregation points for direct
query and analysis of global or enclave data, though it is potentially
less scalable to perform centrally than in a distrubted fashion, depending
on the amount of data being processed by the system.
%
%
\par
%
SOSflow's internal data processing utilizes unbounded asynchrous queues
for all messaging, aggregation, and data storage.
%
Pervasive design around asynchronous data movement allows for the SOSflow
runtime to efficiently handle requests from clients and messaging between
off-node daemons without incurring synchronization delays.
%
Asynchronous in situ design allows the SOSflow runtime to scale out
beyond the practical limits imposed by globally synchronous data movement
patterns.
%
\subsubsection{Aggregate} %----------%
%
A global perspective on application and system performance is often useful.
%
SOSflow automatically migrates information it is given into one or more
aggregation targets.
%
This movement of information is transparent to user's of SOS, requiring
no additional work on their part.
%
Aggregation targets are fully-functional instances of the SOSflow daemon,
except that their principle data sources are the distributed in situ daemons
rather than local clients.
%
The aggregated data contains identical information as the in situ data stores,
it just has more of it, and it is assembled into one location.
%
The aggregate daemons are useful for performing online analysis or information
visualization that needs to include information from multiple nodes.
%
%
\par
%
SOSflow is not a publish-subscribe system in the traditional sense, but
uses a more scalable push and pull model.
%
Everything sent into the system will automatically migrate to
aggregation points unless it is explicitly tagged as being node-only.
%
Requests for information from SOSflow are ad hoc and the scope of the
request is constrained by the location where the request is serviced:
in situ queries are resolved against the in situ database, aggregate
queries are resolved against the aggregate database.
%
If the node-only information is potentially useful for offline analysis
or archival, the in situ data stores can be collected at the end of
a job script, and their contents can be filtered for that node-only
information, which can be simply concatenated together with the aggregate
database[s] into a complete image of all data
%
All information in SOSflow is tagged with a globally unique identifier
(GUID).
%
This allows SOSflow data to be mixed together while preserving its provenance
and preventing data duplication or namespace collision.
%
% ----------------------------------------------------------------------------%
\subsection{SOSflow Client Library}
%
Clients directly interface with the SOSflow runtime system by calling a 
library of functions (libsos) through a standardized API.
%
All communication between the SOSflow library and daemon are transparent
to users, they do not need to write any socket code or introduce any
state or additional complexity to their own code.
%
%
\par
\todo[inline]{TABLE OF INIT/PACK/PUB API FUNCTION SIG.S W/DESCRIPTIONS.}
%
SOSflow forms its own communication channels, and does not interfere
with or rely on any communication overlay formed by the application it is
linked into.
%
\begin{figure}[h]
\centering
\includegraphics[width=\columnwidth]{images/placeholder.jpg}
\caption{SOSflow client library.}
\label{fig_client_library}
\end{figure}
%
Information sent through the libsos API is copied into internal data
structures, and can be freed or destroyed by the user after the SOSflow
API function returns.
%
Data provided to the API is published up to the in situ daemon with
an explicit API call, allowing developers to control the frequency
of interactions with the runtime environment.
%
It also allows the user to register callback functions that can be
triggered and provided data by user-defined analytics function,
creating an end-to-end system for both monitoring as well as feedback
and control.
%
%
\par
%
To maximize compatibility with extant HPC applications, the SOSflow client
library is presently implemented in C99.
%
The use of C99 allows the library to be linked in with a wide variety of
HPC application codes, performance tools, and operating environments.
%
There are various custom object types employed by the SOSflow API, and these
custom types can add a layer of complexity when binding the full API to
a language other than C or C++.
%
SOSflow provides a solution to this challenge by offering a "Simple SOS"
(ssos) wrapper around the full client library, exposing an API that uses no
custom types.
%
The ssos wrapper was used to build a native Python module for SOSflow.
%
Users can directly interact with the SOSflow runtime environment from
within Python scripts, acting both as a source for data, but also a
consumer of online query results.
%
SOSflow users can extend the utility of their applications using
\textbf{numpy} for analytics and \textbf{pyplot} for information
visualization, as a simple example.
%
HPC developers can capitalize on the ease of development provided by Python,
using SOSflow to observe and react to online data from complex applications
written in legacy languages and data models.
%
% ----------------------------------------------------------------------------%
\subsection{SOSflow Data}
%
The primary concept around which SOSflow organizes information is the
"publication handle" (pub).
%
Pubs provide a private namespace where many types and quantities of
information can be stored as a key/value pair.
%
SOSflow automatically annotates values with a variety of metadata,
including a GUID, timestamps, origin application, node id, etc.
%
This metadata is available in the persistent data store for online query
and analysis.
%
SOSflow's metadata is useful for a variety of purposes:
\begin{itemize}
    \item Performance analysis
    \item Provenance of captured values for detection of
        source-specific patterns of behavior, failing
        hardware, etc.
    \item Interpolating values contributed from multiple
        source applications or nodes
    \item Re-examining data after it has been gathered, but
        organizing the data by metrics other than those
        originally used when it was gathered
\end{itemize}
%
%
\par
%
The full history of a value, including updates to that value, is maintained
in the daemon's persistent data store.
%
When a key is re-used to store some new information that has not yet been
transmitted to the in situ daemon, the client library enqueues it up as
a snapshot of that value, preserving all associated metadata alongside the
historical value.
%
The next time the client publishes to the daemon, the current and all
enqueued historical values are transmitted.
%
\begin{figure}[h]
\centering
\includegraphics[width=\columnwidth]{images/placeholder.jpg}
\caption{SOSflow data store.}
\label{fig_data_store}
\end{figure}
%
SOSflow is built on a model of a global information space.
%
Aggregate data stores are guaranteed to provide eventual consistency with
the data stores of the in situ daemons that are targeting them.
%
SOSflow's use of continuous but asynchronous movement of information through
the runtime system does not allow for strict quality-of-service guarantees
about the timeliness of information being available for analysis.
%
A parametric scaling study is included in this paper that demonstrates the
performance that can reasonably be expected on a real-world HPC cluster.
%
%
\par
%
SOSflow does not require the use of a domain-specific language when
pushing values into its API.
%
Pubs are self-defining through use: When a new key is used to pack a value
into a pub, the schema is automatically updated to reflect the name and the
type of that value.
%
When the schema of a pub changes, the changes are automatically announced
to the in situ daemon the next time the client publishes data to it.
%
Once processed and injected into the persistent data store, values and their
metadata are accessible via standardized SQL queries.
%
SOSflow's online resolution of SQL queries provides a high-degree of
programmability and online adaptivity to users.
%
Examples and further documentation of SOSflow's SQL schema are provided
within the standard SOSflow repository.
%
%
% ----------------------------------------------------------------------------%
\subsection{SOSflow Analytics \& Feedback}
%
%
\par
%
\todo[inline]{INSERT SOURCE CODE EXAMPLE W/CALLBACK REGISTRATION/TRIGGERING}
%
\begin{figure}[h]
\centering
\includegraphics[width=\columnwidth]{images/placeholder.jpg}
\caption{SOSflow analytics interface.}
\label{fig_analytics_interface}
\end{figure}
%
%
%
%%%%%%%%%%%%%%%%%%%%%%%%%%%%%%%%%%%%%%%%%%%%%%%%%%%%%%%%%%%%%%%%%%%%%%%%%%%%%%%
%%%%%%%%%%%%%%%%%%%%%%%%%%%%%%%%%%%%%%%%%%%%%%%%%%%%%%%%%%%%%%%%%%%%%%%%%%%%%%%
%%%
%%%  EOF
%%%
